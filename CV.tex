% Inspiration: https://github.com/sebastianruder/cv and https://github.com/LukasSchaefer/CV

\documentclass[10pt,letterpaper]{article}

\usepackage[letterpaper,margin=0.5in]{geometry}
%\usepackage[utf8]{inputenc} % Does not work with current code, therefore inputenc basic
\usepackage{inputenc}
\usepackage{mdwlist}
\usepackage[T1]{fontenc}
\usepackage{textcomp}
%\usepackage{tgpagella} % Does not work with current code
\usepackage{latexsym}
\usepackage{amssymb}
\usepackage[hidelinks]{hyperref}
%\usepackage{fontawesome}

\pagestyle{empty}
\setlength{\tabcolsep}{0em}

% indentsection style, used for sections that aren't already in lists
% that need indentation to the level of all text in the document
\newenvironment{indentsection}[1]%
{\begin{list}{}%
	{\setlength{\leftmargin}{#1}}%
	\item[]%
}
{\end{list}}

% opposite of above; bump a section back toward the left margin
\newenvironment{unindentsection}[1]%
{\begin{list}{}%
	{\setlength{\leftmargin}{-0.5#1}}%
	\item[]%
}
{\end{list}}

% format two pieces of text, one left aligned and one right aligned
\newcommand{\headerrow}[2]
{\begin{tabular*}{\linewidth}{l@{\extracolsep{\fill}}r}
	#1 &
	#2 \\
\end{tabular*}}



\begin{document}

\begin{center}
{\LARGE \textbf{Jonas Schäfer}}
\\
\ \\
Room 1, Flat 116, Chamberlain, 37E Church Road \ - \ B15 3SZ, Birmingham \ - \ United Kingdom
\\
\texttt{jonas.schaefer00@gmail.com} \ \textbullet \ \ \href{tel:+447542546497}{+44 7542 546497}‬ \ \textbullet \ \ \href{https://www.linkedin.com/in/jonas-schaefer/}{www.linkedin.com/in/jonas-schaefer}
\end{center}

\subsection*{Education}
\hrule
\vspace{0.4em}

\noindent
\headerrow{\textbf{University of Birmingham}}{\textbf{Birmingham, United Kingdom}}
\\
\headerrow{\emph{Bachelor of Science (BSc) in Computer Science}}{\emph{09/2018 -- Present}}
\vspace{-1.6em}
\begin{itemize*}
    \item Expected graduation in June 2021
	\item Current average of assessed work: \textbf{85\footnotesize{\%}}
    \item First year Computer Science courses (20CP each): \emph{Programming in Java, Mathematical Foundations, \\Artifical Intelligence, Data Structures \& Algorithms, Logic \& Computation}
    \item First year Widening Horizons Module - Astronomy (20CP): \emph{The Cosmic Connection}


\end{itemize*}

\noindent
\headerrow{\textbf{Warndt-Gymnasium, Völklingen}}{\textbf{Geislautern, Germany}}
\\
\headerrow{\emph{Secondary School}}{\emph{08/2010 -- 07/2018}}
\vspace{-1.6em}
\begin{itemize*}
	\item Graduated \textbf{Abitur 1.5} with examination subjects:\\
       English - 14, Mathematics - 13, Informatics - 12, Geography - 10,
       German - 13
    \item Honor received for \textbf{Year's best Informatics exam}

\end{itemize*}


\subsection*{Programming Projects}
\hrule
\vspace{0.4em}
\noindent
\headerrow{\textbf{Maze Mapper - Data Structures \& Algorithms - 100/100}}{\emph{01/2019 - 02/2019}}
\vspace{-1.6em}
\begin{itemize}
    \setlength\itemsep{0em}
    \item Implementing a \emph{Drone} class that can move through an arbitrary \emph{Maze} (consisting of chambers and connections between them) and maps it using various data structures and methods. At any point it can return (loop-less) to its origin.
    \item Improved my programming capabilities by using more complex data structures and implementing an advanced project.
\end{itemize}
\vspace{0.4em}

\noindent
\headerrow{\textbf{Genetic Algorithm(s) - Programming in Java - 100/100}}{\emph{11/2018 - 12/2018}}
\vspace{-1.6em}
\begin{itemize}
    \setlength\itemsep{0em}
    \item Designing an \emph{Individual} representation as well as implementing an abstract genetic algorithm \emph{GAApplication} class that can be specified into a \emph{Binary Maximiser}, \emph{Weasel} or \emph{Maths} genetic algorithm.
    \item Learned to properly use tools of Java as an Object-oriented programming language and how to design and adjust GAs.
\end{itemize}
\vspace{0.4em}

\noindent
\headerrow{\textbf{Email-Address Finder - Programming in Java - 95/100}}{\emph{10/2018 - 11/2018}}
\vspace{-1.6em}
\begin{itemize}
    \setlength\itemsep{0em}
    \item Implementation of a \emph{findEmailAddress} method to read valid email addresses out of a corrupted database file
    \item Quickly learned to apply the newly learned programming language theory of \emph{Java} to write this first proper method.
\end{itemize}
\emph{Further projects, details and code can be found on my \href{https://github.com/j0ner0n}{\underline{GitHub repository}}}

\subsection*{Internship Experience}
\hrule
\vspace{0.4em}

\noindent
\headerrow{\textbf{Engineering Internship "IngFo" at Saarland University (2 weeks)}}{\emph{07/2015}}
\\
\vspace{-1.6em}
\begin{itemize*}
    \parskip=0.1em
    \item 2 week Internship introducing me to a wide range of Engineering fields: \\
    \emph{Materials Engineering, Materials Science, Systems Engineering, Automation Engineering, Construction Technology, Metrology, Drive Technology, Micro- and Nanotechnology}
\end{itemize*}

%\vspace{0.4em}

\subsection*{Extracurricular Activities}
\hrule
\vspace{0.4em}

\noindent
\headerrow{\textbf{Astronomy Talk}}{\emph{03/2017}}
\vspace{-1.6em}
\begin{itemize}
    \setlength\itemsep{0em}
    \item Advertising science subjects at the \emph{Science night} of the Warndtgymnasium by explaining astronomic phenomenons
    \item Demonstrated presentation skills and the ability to share enthusiasm for Science with others
\end{itemize}

%\noindent
%\headerrow{\textbf{Free tutoring lessons (Mathematics)}}{\emph{08/2016 - 07/2018}}

%\vspace{0.4em}

\subsection*{Skills \& Interests}
\hrule
\vspace{0.4em}
\begin{description*}
	\item[Programming:]
	Java%, Basics: Pascal, Assembly, Python, C
    	\item[Markup:]
    	\LaTeX, Markdown
	\item[Technologies / Tools:]
	Git
	\item[Languages:]
	German (native), English (fluent), French (advanced), Polish (basic)
   	\item[Interests:]
	Programming, Sciences (esp. Space-related), Music (esp. playing the guitar \& piano), Languages, Travelling
\end{description*}


% \subsection*{References}
% \hrule
% \vspace{0.8em}
% \noindent

\hfill \small \textit{[References available on request - CV last updated as of {\today}]}

\end{document}
